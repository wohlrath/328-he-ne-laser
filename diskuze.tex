\section*{Diskuze}
Při měření profilu svazku jsme zjistili, že mikrometrický šroub použitý k posunu štěrbiny detektoru, měl poměrně vysokou vůli, což mohlo ovlivnit přesnost měření, ikdyž jsme se její vliv snažili eliminovat posunem pouze v jednom směru.
Profil svazku vyšel podle očekávání přibližně dobře gaussovský.

Při měření stupně polarizace jsme používali plastový polarizátor. S lepším polarizátorem bychom možná naměřili vyšší stupeň polarizace, proto naměřenou hodnotu chápeme spíše jako dolní mez. Minimální intenzitu jsme měřili na nejmenším rozsahu na jednu platnou číslici, a proto je určena s poměrně vysokou relativní chybou.

Body v grafu \ref{g:div} (divergence) leží poměrně dobře na přímce a má smysl uvažovat pouze jednu hodnotu divergence.

Vlnová délka vyšla podle očekávání.

Relativní jednotky intenzity jsou v každé tabulce jiné, tzn. nelze porovnávat hodnoty v různých tabulkách.